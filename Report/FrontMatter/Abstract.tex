\begin{abstract}
	{\small
	Presentamos un sistema de generación de textos alternativos para imágenes del repositorio digital de la Oficina del Historiador de La Habana. Se exploraron modelos avanzados de aprendizaje profundo, incluyendo BLIP, Visual Transformers (GPT-2) y CLIP, para generar y seleccionar los textos alternativos. Se realizó un análisis del estado del arte, identificando modelos de redes recurrentes, convolucionales y basados en Transformers. La metodología combinó la generación de descripciones con un algoritmo de selección basado en similitud semántica. Los resultados fueron evaluados mediante métricas BLEU, METEOR y CIDEr, mostrando un buen desempeño. Se concluye que la combinación de modelos mejora la calidad de las descripciones, sentando bases para futuras mejoras.
}
\end{abstract}

%\begin{enabstract}
%         {\small }
%\end{enabstract}