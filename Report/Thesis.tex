\documentclass[11pt,oneside]{uhthesis}
%\documentclass[11pt,oneside]{report}
\usepackage{subfigure}
\usepackage[linesnumbered,lined,titlenumbered,ruled,algochapter,spanish,onelanguage]{algorithm2e}


\usepackage{substr}
\usepackage{bbm}
\usepackage{lipsum}
%\usepackage{easyReview}

\usepackage{amsmath}
\usepackage{theorem}
\usepackage{amssymb}
\usepackage{amsbsy}
%\usepackage{mathpazo}


\usepackage{float}
\usepackage{braket}
\usepackage[htt]{hyphenat}
\setlength {\marginparwidth }{3cm}


\usepackage[spanish]{babel}
\usepackage{graphicx}

\usepackage{listings}
\usepackage{color}
\usepackage{booktabs}
\usepackage{multirow}
\usepackage{ragged2e}
\usepackage{multicol}
\usepackage{comment} 

\usepackage{hyphenat}
\hyphenation{in-te-rac-tú-an}

\spanishdecimal{.}

\floatstyle{ruled}
\restylefloat{table}
\usepackage{color}

\usepackage[disable]{todonotes}

\definecolor{dkgreen}{rgb}{0,0.6,0}
\definecolor{gray}{rgb}{0.2,0,0}
\definecolor{mauve}{rgb}{0.58,0,0.82}

\lstset{language=Python,
	aboveskip=10mm,
	belowskip=10mm,
	showstringspaces=false,
	columns=flexible,
	basicstyle={\small\ttfamily},
	keywordstyle=\color{blue},
	commentstyle=\color{dkgreen},
	stringstyle=\color{mauve},
	breaklines=true,
	breakatwhitespace=true,
	tabsize=3,
	numbers=left, numberstyle=\tiny, stepnumber=1,firstnumber=1,
	numbersep=5pt
}

\renewcommand{\tablename}{Tabla}
\title{Textos alternativos para las colecciones de imágenes del repositorio digital de la Oficina  del historiador de La Habana}
\author{\\Amanda Cordero Lezcano\\ Ana Paula González Muñoz\\ Carlos Antonio Bresó Sotto\\ Christopher Guerra Herrero\\ Dennis Daniel González Durán\\ Marian Susana Álvarez Suri}

\faculty{Facultad de Matemática y Computación}
\date{Enero 2025}
\logo{Graphics/uhlogo}

\renewcommand{\vec}[1]{\boldsymbol{#1}}
\newcommand{\diff}[1]{\ensuremath{\mathrm{d}#1}}


\newcommand{\lachinote}[2][noinline]{\todo[author=Lachi,backgroundcolor=cyan,linecolor=cyan,#1]{\footnotesize {#2}}}

\newcommand{\correccion}[2][noinline]{\todo[backgroundcolor=pink,linecolor=pink,#1]{\scriptsize {#2}}}



% Para definir teoremas, lemas, definiciones, corolarios, pruebas

\newtheorem{definicion}{Definición}
\newtheorem{lema}{Lema}
\theoremstyle{break}
\newtheorem{teorema}{Teorema}
\theoremstyle{break}
\newtheorem{corolario}{Corolario}[teorema]
\newtheorem{prueba}{Prueba}
%\theoremstyle{break}

% Para citar teoremas, lemas, definiciones, corolarios, pruebas
\newcommand{\reftheorem}[1]{\IfBeforeSubStringEmpty{teorema}{#1}{\textbf{Teorema~\ref{#1}}}{\IfBeforeSubStringEmpty{lema}{#1}{\textbf{Lema~\ref{#1}}}{\IfBeforeSubStringEmpty{prueba}{#1}{\textbf{Prueba~\ref{#1}}}{\IfBeforeSubStringEmpty{corolario}{#1}{\textbf{Corolario~\ref{#1}}}{\IfBeforeSubStringEmpty{definicion}{#1}{\textbf{Definición~\ref{#1}}}{{\color{red} Citado Erróneo}}}}}}}

\newcommand{\reftheorempage}[1]{\IfBeforeSubStringEmpty{teorema}{#1}{\textbf{Teorema~\ref{#1}} (pág \pageref{#1})}{\IfBeforeSubStringEmpty{lema}{#1}{\textbf{Lema~\ref{#1}} (pág \pageref{#1})}{\IfBeforeSubStringEmpty{prueba}{#1}{\textbf{Prueba~\ref{#1}} (pág \pageref{#1})}{\IfBeforeSubStringEmpty{corolario}{#1}{\textbf{Corolario~\ref{#1}} (pág \pageref{#1})}{\IfBeforeSubStringEmpty{definicion}{#1}{\textbf{Definición~ \ref{#1}} (pág \pageref{#1})}{{\color{red} Citado Erróneo}}}}}}}

\begin{document}
\selectlanguage{spanish}

%\frontmatter
\maketitle

%\include{FrontMatter/Dedication}
%\include{FrontMatter/Thanks}
%\include{FrontMatter/SupervisorOpinion}
\begin{abstract}
	{\small
	Presentamos un sistema de generación de textos alternativos para imágenes del repositorio digital de la Oficina del Historiador de La Habana. Se exploraron modelos avanzados de aprendizaje profundo, incluyendo BLIP, Visual Transformers (GPT-2) y CLIP, para generar y seleccionar los textos alternativos. Se realizó un análisis del estado del arte, identificando modelos de redes recurrentes, convolucionales y basados en Transformers. La metodología combinó la generación de descripciones con un algoritmo de selección basado en similitud semántica. Los resultados fueron evaluados mediante métricas BLEU, METEOR y CIDEr, mostrando un buen desempeño. Se concluye que la combinación de modelos mejora la calidad de las descripciones, sentando bases para futuras mejoras.
}
\end{abstract}

%\begin{enabstract}
%         {\small }
%\end{enabstract}
\include{FrontMatter/Contents}


%\mainmatter
	
%===================================================================================
% Chapter: Introduction
%===================================================================================
\chapter{Introducción}\label{chapter:introduction}
%===================================================================================


 
%===================================================================================
% Chapter: Estado del Arte
%===================================================================================
\chapter{Estado del Arte}\label{chapter:estadoarte}
%===================================================================================

El campo del \textit{image captioning} ha evolucionado significativamente en la última década, impulsado por el avance de modelos de aprendizaje profundo. En este capítulo, se presentan los principales enfoques y modelos que han marcado hitos en esta área, destacando sus arquitecturas, metodologías y contribuciones.

\section{Modelos Basados en Redes Recurrentes}

Los primeros avances en generación de descripciones de imágenes se apoyaron en arquitecturas \textit{encoder-decoder} con redes neuronales recurrentes (RNN). Uno de los primeros modelos destacados fue \textbf{Show and Tell} \cite{vinyals2015show}, que utilizó una combinación de una red convolucional (CNN) para la extracción de características visuales y una red LSTM para la generación de texto. Este modelo logró buenos resultados en datasets como MSCOCO y Flickr30k, evaluándose con métricas como BLEU y METEOR.

Posteriormente, \textbf{Show, Attend and Tell} \cite{xu2015show} introdujo mecanismos de atención visual, permitiendo que el modelo enfocara diferentes regiones de la imagen en cada paso de generación. Este enfoque mejoró la calidad de las descripciones y presentó una formulación matemática más avanzada para el cálculo de la atención.

\section{Modelos Basados en Redes Convolucionales}

En un intento por superar las limitaciones de las RNN, \textbf{Convolutional Image Captioning} \cite{aneja2018convcap} propuso una arquitectura basada en CNNs para la generación de texto. Este modelo demostró que las CNNs pueden superar a las LSTM en tareas de \textit{captioning}, especialmente cuando se combinan con mecanismos de atención, mitigando problemas como el desvanecimiento del gradiente.

\section{Atención y Modelos Jerárquicos}

Otro enfoque relevante fue el modelo \textbf{Bottom-Up and Top-Down} \cite{anderson2018bottom}, que implementó un mecanismo de atención jerárquico basado en la segmentación de objetos dentro de la imagen. Este modelo mejoró el rendimiento en tareas como \textit{Visual Question Answering} e \textit{image captioning}, utilizando datasets como Visual Genome y MSCOCO.

\textbf{Knowing When to Look} \cite{lu2017knowing} propuso un mecanismo de atención adaptativo mediante un centinela visual, que decide cuándo prestar atención a la imagen y cuándo confiar en el contexto textual generado previamente. Este modelo ofreció mejoras en métricas como CIDEr y BLEU.

\section{Transformers y Modelos Multimodales}

Con la llegada de los Transformers, los modelos de \textit{captioning} han adoptado arquitecturas más avanzadas. \textbf{Multimodal Transformer} \cite{yu2019multimodal} exploró la representación visual multi-vista para mejorar la generación de texto, mientras que \textbf{BLIP} \cite{li2022blip} amplió la capacidad de los modelos al abordar tanto la generación como la comprensión de imágenes y videos, logrando mejoras en tareas como recuperación de imágenes y preguntas visuales.

Además, \textbf{CLIP} \cite{radford2021learning} introdujo el aprendizaje multimodal a gran escala utilizando correspondencias entre imágenes y texto en internet, mientras que \textbf{ViT} \cite{dosovitskiy2021image} aplicó Transformers directamente a imágenes dividiéndolas en parches, lo que representó un cambio significativo en el procesamiento de información visual.

\section{Resumen y Tendencias Actuales}

La evolución del \textit{image captioning} ha pasado de modelos basados en RNNs con atención visual a enfoques más sofisticados que integran Transformers y aprendizaje multimodal. Modelos recientes como BLIP y CLIP han demostrado que la combinación de visión y lenguaje en grandes volúmenes de datos puede llevar a mejoras sustanciales en la generación y comprensión de imágenes. 

Las tendencias actuales apuntan a modelos más eficientes y escalables, con capacidades mejoradas en la generación de texto y una mayor comprensión del contexto visual. El impacto de estas tecnologías se extiende más allá del \textit{image captioning}, beneficiando tareas como búsqueda visual, generación de contenido y asistencia en accesibilidad.



%===================================================================================
% Chapter: Introduction
%===================================================================================
\chapter{Propuesta}\label{chapter:propuesta}
%===================================================================================
Se propone un sistema para la generación de texto alternativo a partir de las imágenes del repositorio digital del patrimonio cultural de la Oficina del Historiador, combinando múltiples modelos de aprendizaje profundo. La metodología se basa en un enfoque de evaluación comparativa entre diferentes modelos de generación de texto a partir de imágenes, utilizando un criterio de selección basado en la similitud semántica con el contenido visual.

En primer lugar, se emplea el modelo BLIP (Bootstrapped Language-Image Pretraining) para generar una primera descripción de la imagen. Posteriormente, la imagen es procesada por el modelo ViT (Vision Transformer) en conjunto con GPT-2, obteniendo una segunda descripción independiente. Finalmente, el modelo CLIP (Contrastive Language-Image Pretraining) se utiliza como mecanismo de selección, comparando las dos descripciones generadas y eligiendo la que presente una mayor correspondencia semántica con la imagen de entrada.

% TODO: \usepackage{graphicx} required
\begin{figure}
	\centering
	\includegraphics[width=1\linewidth]{./Graphics/diagrama}
	\caption{Modelo general}
	\label{fig:diagrama}
\end{figure}

%===================================================================================
% Chapter: Introduction
%===================================================================================
\chapter{Desarrollo de la propuesta}\label{chapter:desarrollo}
%===================================================================================
El primer paso en nuestro proyecto fue la descarga del dataset de imágenes desde el servidor.

Se realizó una investigación exhaustiva sobre el estado del arte en el campo de procesamiento de imágenes para generación de textos alternativos. El resumen incluyó una caracterización de las formas de aprendizaje en los modelos de Machine Learning, la arquitectura que se llevó a cabo para el desarrollo del algoritmo, el modelo de lenguaje empleado, así como los datasets y métricas que fueron seleccionados durante el entrenamiento y evaluación de los modelos. 

Se seleccionaron los modelos que mejor se adaptaban a nuestro caso, optando por BLIP y Visual Transformers (GPT-2), los cuales destacan por su eficacia en tareas de procesamiento de imágenes y generación de texto. 

Se realizó un análisis exploratorio de los datos para comprender mejor las características y peculiaridades del dataset, lo cual contribuyó a identificar patrones y posibles problemas que podrían ser enfrentados durante el procesamiento de las imágenes.

Se procesaron todas las imágenes localmente utilizando los modelos seleccionados y mediante el uso de CLIP, se seleccionó la descripción que mejor se adaptaba a cada imagen.

Utilizamos las métricas BLEU y METEOR sobre los resultados obtenidos para evaluar el rendimiento de nuestros modelos, lo que nos permitió medir la efectividad de nuestro enfoque.

Finalmente, analizamos estadísticas básicas sobre los resultados finales. Este análisis nos proporcionó una visión clara de la calidad y consistencia de nuestras descripciones generadas y nos ayudó a identificar áreas de mejora para futuros proyectos.
%\include{MainMatter/EstimacionParametros}
%\include{MainMatter/Habana}
%\include{MainMatter/ModelosClasicos}

%===================================================================================
% Chapter: Introduction
%===================================================================================
\chapter{Resultados}\label{chapter:resultados}
%===================================================================================

Para evaluar nuestro modelo de captioning de imágenes, seleccionamos cuidadosamente un conjunto de 391 imágenes provenientes de diversas colecciones, garantizando una amplia representación del dataset.

Una vez que las imágenes fueron seleccionadas, nustro equipo creó captions descriptivas para cada una de ellas. Estas captions humanas fueron compiladas en un documento, que sirvió como conjunto de referencia para la evaluación.

Con el documento de captions humanas listo, procedimos a utilizar nuestro modelo de captioning para generar descripciones automáticas para las mismas 391 imágenes. Estas captions generadas por el modelo fueron luego comparadas directamente con las captions humanas utilizando métricas de evaluación estándar como BLEU, METEOR, CIDEr y SPICE.

Métricas usadas:

- CIDEr: la métrica CIDEr (Consensus-based Image Description Evaluation) es una forma de evaluar la calidad de las descripciones textuales generadas de imágenes. La métrica CIDEr mide la similitud entre una caption generada y las captions de referencia, y se basa en el concepto de consenso: la idea de que las buenas captions no solo deben ser similares a las captions de referencia en términos de elección de palabras y gramática, sino también en términos de significado y contenido.

- SPICE:
La métrica SPICE (Semantic Propositional Image Caption Evaluation) mide la calidad semántica de las descripciones de imágenes generadas por modelos. A diferencia de otras métricas que se centran en la similitud superficial de palabras, SPICE evalúa el contenido semántico dividiendo las descripciones en representaciones de gráficos semánticos, que incluyen objetos, atributos y relaciones. Estos gráficos se comparan luego con gráficos de referencia creados a partir de descripciones humanas. La métrica SPICE se enfoca en capturar la precisión y la integridad semántica, asegurándose de que la descripción generada refleje correctamente los elementos y las relaciones presentes en la imagen.

Los valores de la métrica SPICE se interpretan en términos de precisión semántica y correspondencia con las descripciones humanas. Un valor SPICE alto indica que la descripción generada por el modelo representa de manera precisa y completa los elementos y relaciones de la imagen, similar a como lo harían los humanos. Por el contrario, un valor SPICE bajo sugiere que la descripción carece de precisión o no captura adecuadamente el contenido semántico de la imagen.

- BLEU(Bilingual Evaluation Understudy): es un algoritmo utilizado para evaluar la calidad del texto que ha sido traducido automáticamente de un idioma natural a otro. Fue inventado en IBM en 2001 y es una de las primeras métricas en afirmar una alta correlación con los juicios humanos de calidad.

El puntaje BLEU se calcula comparando el texto traducido automáticamente (candidato) con uno o más textos traducidos profesionalmente por humanos (referencias). Se considera que la calidad es la correspondencia entre la salida de la máquina y la de un humano. La idea central detrás de BLEU es que "cuanto más cercana sea una traducción automática a una traducción profesional humana, mejor será".


- Meteor: es una métrica utilizada para evaluar la traducción automática comparándola con traducciones humanas. Tiene en cuenta tanto la precisión como la fluidez de la traducción, así como el orden en que aparecen las palabras. El puntaje METEOR varía de 0 a 1, con un puntaje más alto indicando mejor calidad de traducción.

El algoritmo detrás del puntaje METEOR compara el texto traducido con la traducción de referencia humana descomponiéndolos en fragmentos y calculando la similitud entre cada fragmento utilizando varias medidas, como la precisión de unigramas, el recall y el F-score, la superposición de bigramas y las coincidencias exactas de palabras. Finalmente, se utiliza el promedio ponderado de estas medidas para calcular el puntaje METEOR general.




%\backmatter
%===================================================================================
% Chapter: Conclusiones
%===================================================================================
\chapter{Conclusiones}\label{chapter:conclusions}

La evolución de los modelos de generación de descripciones de imágenes ha sido notable en los últimos años, pasando de arquitecturas basadas en RNNs con atención visual a enfoques más avanzados que integran Transformers y aprendizaje multimodal.Modelos recientes, como CLIP y BLIP, han demostrado el potencial de combinar grandes volúmenes de datos de texto e imagen, logrando una comprensión más profunda del contenido visual. Estos avances no solo han mejorado la calidad y precisión de las descripciones generadas, sino que también han abierto nuevas posibilidades en tareas de visión y lenguaje, estableciendo el camino para futuras investigaciones en la intersección entre inteligencia artificial y percepción visual.

Los resultados obtenidos en la evaluación del modelo presentado evidencian que, si bien el sistema logra generar descripciones con coherencia léxica, presenta limitaciones en la captura de relaciones semánticas profundas.  Sin embargo, es importante tener en cuenta que la generación de subtítulos manualmente se limita a un conjunto específico de imágenes y no fueron realizadas por expertos.
Al no contar con un método supervisado previamente, la evaluación de los resultados constituyó un desafío.
Los puntajes moderados en METEOR y ROUGE reflejan que el modelo mantiene cierta similitud con las descripciones humanas, lo que implica que la generación de texto sigue patrones lingüísticos aceptables, aunque con margen de mejora en la exactitud del contenido.    

%\include{BackMatter/Recommendations}
\renewcommand{\bibname}{Referencias}
% \nocite{*}
\bibliographystyle{unsrt}
\bibliography{Bibliography.bib}
%\include{BackMatter/References}
%\include{BackMatter/Glossary}


\end{document}