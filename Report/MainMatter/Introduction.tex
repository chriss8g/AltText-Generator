%===================================================================================
% Chapter: Introduction
%===================================================================================
\chapter{Introducción}\label{chapter:introduction}
%===================================================================================
En este proyecto, se desarrolla un sistema de generación de texto alternativo para las imágenes del repositorio digital de la Oficina del Historiador de la Ciudad de La Habana. Creando descripciones precisas y detalladas que mejoren la accesibilidad a la información, permitan una mejor búsqueda y recuperación de imágenes. 

El problema de generar texto alternativo para imágenes ha sido abordado mediante el uso de técnicas avanzadas de procesamiento del lenguaje natural (NLP) y visión por computadora. Investigaciones recientes han utilizado redes neuronales convolucionales (CNN) para analizar y comprender el contenido visual de las imágenes, combinadas con redes neuronales recurrentes (RNN) o Transformers para generar descripciones textuales coherentes y contextualmente adecuadas. Estos modelos han demostrado ser efectivos en la tarea de generación de texto alternativo, permitiendo la creación automática de descripciones.

La propuesta de solución implica la combinación de dos modelos pre-entrenados de generación de texto alternativo, con los cuales se generan dos descripciones para cada imagen, y luego se aplica un algoritmo de selección que elige la descripción que mejor se ajusta a cada imagen específica. Esta metodología no solo aprovecha las capacidades avanzadas de cada modelo, sino que también asegura que la descripción final sea la más precisa y relevante.